\documentclass[12pt]{scrartcl}
\usepackage{ucs}
\usepackage[utf8x]{inputenc}
\usepackage[english,ngerman]{babel}
\usepackage{amsmath,amssymb,amstext}
\usepackage{graphicx}
\usepackage[automark]{scrpage2}
\usepackage{txfonts} 
\pagestyle{scrheadings}

\ifoot[Voll]{Agil}
\title{Sprintplanung}
\author{Philipp Koch}
\date{\today{}, Nuremberg}

\begin{document}
\maketitle

\section{Sprintlänge}
Hier könnten wir noch darüber reden, ob wir die Sprints immer gleich lang machen, oder ob wir das von der Menge der Arbeit abhängig machen, die bis zum nächsten Spring erledigt sein muss.

\textbf{Attention:} Genaueres nachlesen auf Seite 34 im "Skript" sagt an, dass die Sprints 2 -4 Wochen lang sein müssen und am Ende ein Release enthalten. Ich würde mal sagen, dass wir (da wir nur 1 Woche haben) die Sprints auf 1 - 2 Tage setzen und als Ziel ein einzelnes Stück Software abliefern.
\section{Sprints}
Laut Skript (s.34) werden vor jedem Sprint die höchstprioren Inhalte des Product Backlogs (BL) ausgewählt und im Sprint bearbeitet. Daher werde ich hier nicht immer das kplt. BL auflisten sondern die beiden höchstprioren Einträge (das Scrum-Team ist zu zweit). Diese wurden natürlich mit dem Product Owner abgestimmt... 
\subsection{Hauptmenu}
\textbf{Product Backlog} (Priorität absteigend)
\begin{enumerate}
\item Hauptmenu-Design (Widget)
\item Hauptmenu-Grundfunktionalität (Dummy-Funktionen)
\end{enumerate}
\subsection{Kartenwidgets}
\textbf{Product Backlog} (Priorität absteigend)
\begin{enumerate}
\item TH-Nuernberg Lageplan integrieren
\item Google-Maps plugin
\end{enumerate}
\subsection{Kommunikation}
\textbf{Product Backlog} (Priorität absteigend)
\begin{enumerate}
\item Schittstelle zu Th-Email und Kalender
\item Design des Kommunikationswidget
\end{enumerate}
\subsection{Kartenfunktionen}
\textbf{Product Backlog} (Priorität absteigend)
\begin{enumerate}
\item Highlighting wichtiger Orte
\item Implementierung interaktiver Liste (User kann selbst Orte hinzufügen)
\end{enumerate}
\textbf{Attention: Erste Frechheit}: Der Name der Hochschule wird geändert
\subsection{Redesign}
\textbf{Product Backlog} (Priorität absteigend)
\begin{enumerate}
\item Ändern der Schriftzüge in Widgets und Startbildschirm
\item Ändern des Logos in Widgets und Startbildschirm
\end{enumerate}







\end{document}