\documentclass[12pt]{scrartcl}
\usepackage{ucs}
\usepackage[utf8x]{inputenc}
\usepackage[english,ngerman]{babel}
\usepackage{amsmath,amssymb,amstext}
\usepackage{graphicx}
\usepackage[automark]{scrpage2}
\usepackage{txfonts} 
\pagestyle{scrheadings}

\ifoot[Voll]{Agil}
\title{Sprintplanung}
\author{Philipp Koch}
\date{\today{}, Nuremberg}

\begin{document}
\maketitle

\section{Sprintlänge}
Hier könnten wir noch darüber reden, ob wir die Sprints immer gleich lang machen, oder ob wir das von der Menge der Arbeit abhängig machen, die bis zum nächsten Spring erledigt sein muss.

\textbf{Attention:} Genaueres nachlesen auf Seite 34 im "Skript" sagt an, dass die Sprints 2 -4 Wochen lang sein müssen und am Ende ein Release enthalten. Ich würde mal sagen, dass wir (da wir nur 1 Woche haben) die Sprints auf 1 - 2 Tage setzen und als Ziel ein einzelnes Stück Software abliefern.
\section{Sprints}
Laut Skript (s.34) werden vor jedem Sprint die höchstprioren Inhalte des Product Backlogs (BL) ausgewählt und im Sprint bearbeitet. Daher werde ich hier nicht immer das kplt. BL auflisten sondern die beiden höchstprioren Einträge (das Scrum-Team ist zu zweit). Diese wurden natürlich mit dem Product Owner abgestimmt... 
\subsection{Hauptmenu}
\textbf{Product Backlog} (Priorität absteigend)
\begin{enumerate}
\item Design des Hauptmenu
\item Design des Startbildes
\item Implementierung der Grundfunktionalitaet (Dummy-Funktionen) 
\end{enumerate}
\subsection{Kartenwidgets}
\textbf{Product Backlog} (Priorität absteigend)
\begin{enumerate}
	\item Lageplan der TH integrieren
	\item Google-maps plugin implementieren
	\item Implementierung interaktiver Liste für Kartenwidgets
	\item Darstellen von Orten aus der Liste in der Karte 
\end{enumerate}
\subsection{Kommunikation}
\textbf{Product Backlog} (Priorität absteigend)
\begin{enumerate}
	\item Schnittstelle zu TH-Email / Kalender implementieren
	\item Reminder in Statusleiste implementieren
	\item Popup für akute Termine implementieren 
	\item Implementierung des Checklistenwidgets
	\item Design des Reminders erstellen
	\item Design des Popups erstellen
	\item Design des Checklistenwidgets erstellen 
\end{enumerate}
\textbf{Attention: PO Änderung}: Der Name der Hochschule wird geändert
\subsection{Redesign}
\textbf{Product Backlog} (Priorität absteigend)
\begin{enumerate}
	\item Änderung der Logos in allen Widgets
	\item Änderung der TH-Schriftzüge in allen Widgets
	\item Redesign des Startbildschirms 
\end{enumerate}







\end{document}