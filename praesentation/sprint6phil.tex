%Sprint 6 user Story & Ablauf -> Phil
\frame{
\frametitle{Sprint 6 Partnerhochschulen - User Stories}
  \begin{block}{User Stories}
   \begin{itemize}
   \item Story 1 - Variables Widget- und Startbildschirmdesign
   
   Logos, Schriftzüge und das Design in den Widgets und im Startbildschirm müssen sich an die Logos unserer Partnerhochschulen anpassen lassen.
   
   \item Story 2 - Variable Datenbankanbindung
   
   Die Datenbankanbindung muss sich an verschiedene THs anpassen lassen.
   
   \item Story 3 - Variabler Hochschullageplan
   
   Der Hochschullageplan muss sich an die entsprechende Hochschule anpassen lassen.
  \end{itemize}
 \end{block}
}
   
\frame{
\frametitle{Sprint 6 Partnerhochschulen - User Stories}
  \begin{block}{User Stories}
   \begin{itemize}   
   \item Story 4 - Automatische Anpassung der App
   
   Beim erstmaligen Starten der App soll der Studierende seine Hochschule aus einer Liste aller teilnehmenden Hochschulen auswählen und die App sich automatisch an die entsprechende Hochschule anpassen.
   \end{itemize}
 \end{block}
}

\frame{
\frametitle{Sprint 6 Partnerhochschulen - Planung}
  \begin{block}{Extraktion aus Product Backlog}
    \begin{itemize}
    \item Variables Widget- und Startbildschirmdesign
    \item Variable Datenbankanbindung
    \item Variabler Hochschullageplan
    \item Automatische Anpassung der App
    \end{itemize}
  \end{block}  
  
}


%\item Story 1 - Aenderung der Logos
%   
%   
%   \item Story 2 - Aenderung der TH-Schriftzuege
%   
%   
%   \item Story 3 - Redesign des Startbildschirms und aller Widgets
%   
%   Der Startbildschirm und alle Widgets der App sollen sich dem Design der Homepage der jeweiligen Partnerhochschule anpassen lassen.